\documentclass{article}
\usepackage{geometry}
 \geometry{
 a4paper,
 total={170mm,257mm},
 left=20mm,
 top=20mm,
 }
\usepackage{graphicx} % Required for inserting images
\usepackage{underscore}
\usepackage{amsmath}

\title{Math 438 Project}
\author{Patrick Beal, Tiara Eddington, TJ Hart, Madelyn Vines, Adam Ward}
\date{March 24, 2025}

\begin{document}

\maketitle

\section{Problem Description}
For our project we decided to model ''Lunar Lander'', a popular game from the 70's. The objective is to move a rocket from one area to another without running into the landscape. The player controls the rocket with thrust, moving the rocket forward, and the angle, which points the rocket in the desired direction. 

\section{Functional}
The main problem we are trying to solving is object avoidance in a setting with multiple moving obstacles. The motivation behind our functional is explained thoroughly in the Object Avoidance lab, with certain adjustments made to better represent the control and constraints we are dealing with. We represent obstacles with a function $$C(x(t), y(t)) = \frac{1}{((x - c_x(t))^2 / r_x + (r - c_y(t))^2 / r_y)^\lambda + 1}.$$ This function defines obstacles as ellipses, with $r_x, r_y$ defining the size, $\lambda$ being a vanishing rate, and $c(t)$ a function specifying how the position of the obstacle's center moves over time.
Now, we want to compare solutions when considering a discrete control space versus a continuous one in the context of guiding a lunar lander with obstacles. For a discrete control caricature, we might restrict admissible actions at each time to a small set (e.g., thrust left/right/off), which is helpful for certain algorithmic comparisons. For simplicity, we can consider a single obstacle at a time and construct a cost that penalizes proximity to obstacles, encourages shorter distances to a target site $x_f$ and shorter overall time $t_f$, and modestly rewards progress when unobstructed. One such discrete-time surrogate objective is: $$\int_{0}^{t_i} (W_1C_i + W_2(x_f - x(t))^2 + W_3u)\,dt + W_4 t_f,$$ where the $W_i$'s are weights.
\\ Turning to the continuous case (our primary model), the control is the lander’s acceleration with bounds $U_l \leq u(t) \leq U_u$ for all $t$. We then use the continuous optimal control formulation described in the main writeup, with obstacle penalties added inside the integral term.

\section{Tools}
While coding up the solution to this problem we will use many Python libraries such as SciPy's \texttt{solve\_bvp}, NumPy, Matplotlib, and potentially others. We hope to use images from the game to create a realistic background for our game and animate it. We are using inspiration from our obstacle avoidance lab.


\end{document}


